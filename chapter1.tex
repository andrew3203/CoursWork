\section{Общие понятия линейной и нелинейной опитки. Нелинейная поляризация}
\subsection{Материальное уравнение линейной среды.}
В основе взаимодействия света со средой лежит элементарный
процесс возбуждения атома или молекулы вещества световым полем и последующего переизлучения света возбужденной частицей. Характер этого взаимодействия зависит от соотношения между величиной напряженности поля световой волны Е и характерной напряженностью внутриатомного поля $E_{atom}$, определяющего силы связи оптических электронов (т.е. внешних, наиболее слабо связанных электронов) с ядром атома вещества.
\\
Для атома водорода это поле составляет $ E_{atom} = e/(4\pi\epsilon_{0}r_{h}^2) = 5\cdot10^{11} $В/м, для более тяжелых атомов $ E_{atom} = 10^{10} \dots 10^{11} $ В/м. Оценка поля Е световой волны в случае нелазерных источников свет дает величину $E \le 10^3$ В/м, т.е. $E<<E_{atom} $. При этом условии отклик атомного осциллятора на внешнее воздействие будет иметь линейный характер, а зависимость поляризованности Р = Р(Е) в случае изотропной среды может быть представлена в виде:

\begin{equation}\label{1:liner}
P(E) = \chi^{(1)}E
\end{equation}

где $ \chi^{(1)}$ – линейная восприимчивость среды, являющаяся безразмерной величиной и зависящая только от свойств среды.
Материальное уравнение ($\ref{1:liner}$) является одним из соотношений, на которых базируется линейная оптика.  Оно справедливо только при условии $E << E_{atom} $, а при невыполнении этого условия является лишь некоторым приближением.В мощных лазерных пучках можно получить напряженности уже сравнимые с $E_{atom} $. В случае когда поле Е, оставаясь меньше $E_{atom} $, приближается к нему по величине, поляризованность среды Р = Р(Е) перестает быть линейной функцией поля Е, и в этом случае материальное уравнение ($\ref{1:liner}$) должно быть заменено на другое.

\subsection{Материальное уравнение нелинейной среды.} 
Теория нелинейно-оптических явлений строится на основе материальных уравнений и уравнений
Максвелла. Уравнения Максвелла для диэлектрической нейтральной немагнитной среды имеют вид

\begin{equation}\label{1:maxvel}
rot\vec{E} = - \frac{ 1 }{ c}\frac{\partial \vec{H} }{\partial t}
\hspace{20mm}
rot\vec{H} =  \frac{ 1 }{ c}\frac{\partial \vec{D} }{\partial t}
\hspace{20mm}
div\vec{H} = 0
\end{equation}

где $ \vec{D} = \vec{E} + 4\pi \vec{P}$. Из уравнений Максвелла вытекает волновое уравнение

\begin{equation}\label{1:rot_maxvel}
rot(rot\vec{E}) + \frac{ 1 }{ c^2 }\frac{\partial^2 \vec{E} }{\partial t^2} = - \frac{ 4\pi }{ c^2 }\frac{\partial^2 \vec{P} }{\partial t^2}
\end{equation}

которе в случае изотропной среды принимает вид

\begin{equation}\label{1:wave_eq}
\Delta\vec{E} - \frac{ 1 }{ c^2 }\frac{\partial^2 \vec{E} }{\partial t^2} =  \frac{ 4\pi }{ c^2 }\frac{\partial^2 \vec{P} }{\partial t^2}
\end{equation}

где $\vec{E}$ - напряженность электрического поля, а $\vec{P}$ - поляризация среды. Поляризация среды возникает под действием падающий световой волны и описывается материальным уравнением $\vec{P} = \vec{P}(\vec{E})$.
В анизотропном случае $\vec{P}(\vec{E})$ являеться тензорный величиной и может быть представлена в виде:
\begin{equation}\label{1:p_1}
P(E) = \chi^{(1)}E + \chi^{(2)}E^2 \chi^{(3)}E^3\dots
\end{equation}

Коэффициенты $\chi^{m}, m \ge 2$ при членах разложения называются нелинейными восприимчивостями m-го порядка и являются уже размерными величинами. При этом соответствующая величина $\chi^{m}$ пропорциональна концентрации атомов (молекул) в веществе и m-ой степени параметра. Это означает, что отклик среды на действие внешнего светового поля перестает быть линейным.  С математической точки зрения именно это обстоятельство (нелинейность материального уравнения) является причиной нарушения принципа суперпозиции для световых волн в нелинейной среде. Из уравнений (\ref{1:rot_maxvel}), (\ref{1:wave_eq}) и (\ref{1:p_1}) непосредственно вытекает возможность генерации оптических гармоник и других нелинейно-оптических эффектов. 
\subsection{Нелинейная поляризация.} 
Часть поляризации среды, нелинейно зависящая от напряженности светового поля, называется нелинейной поляризацией.
Выделяя в поляризации среды линейную и нелинейную компоненты, можно
записать:
\begin{equation}\label{1:p_non_liner}
\vec{P} = \vec{P}_{liner} + \vec{P}_{nonliner}
\end{equation}
подставив уравнение  (\ref{1:p_non_liner}) в  (\ref{1:rot_maxvel}) получим волновое уравнение для анизотропной среды и нелинейной изотропной среды:

\begin{equation}\label{1:wave_eq2}
\Delta\vec{E} - \frac{ 1 }{ c^2 }\frac{\partial^2 \vec{E} }{\partial t^2} - \frac{ 4\pi }{ c^2 }\frac{\partial^2 \vec{P}_{liner}}{\partial t^2} =  \frac{ 4\pi }{ c^2 }\frac{\partial^2 \vec{P} _{nonliner}}{\partial t^2}
\end{equation}

 Нелинейная поляризация среды является источником новых спектральных компонент поля
(оптических гармоник, комбинационных частот и т. п.). Материальное уравнение вида  (\ref{1:wave_eq}), описывает изотропную нелинейную среду с безынерционным локальным откликом на световое поле. Аналогичное
уравнение для анизотропной нелинейной диспергирующей среды имеет уже интегральный вид, причем коэффициенты восприимчивости, входящие в данное уравнение, уже зависят от времени о координаты следующим образом:

 \begin{equation}\label{1:chi}
\chi_{\alpha \beta} =  \chi_{\alpha \beta}(\tau ; \vec{r}),
\hspace{5mm}
\chi_{\alpha\beta\gamma}^{(2)}= \chi_{\alpha\beta\gamma}(\tau_{1},\tau_{2};\vec{r_{1}}, \vec{r_{2} }),
\hspace{5mm}
\chi_{\alpha\beta\gamma\delta}^{(3)} = \chi_{\alpha\beta\gamma\delta}(\tau_{1},\tau_{2},\tau_{3};\vec{r_{1}}, \vec{r_{2}}, \vec{r_{3} })
\dots
\end{equation}
Здесь индексы $\alpha, \beta, \gamma, \dots$ пробегают значения, нумерующие декартовы оси координат. Учет нелокальности важен втех случаях, когда элементарные осцилляторы среды, расположенные в различных точках пространства, связаны и взаимодействуют между собой. Среды, обладающие таким свойством, называют средами с пространственной дисперсией. К их числу относятся некоторые типы кристаллов, а также плазма.
\section{Оптический резонанс полупроводниковых наноструктур}


\section{Основные процессы и эффекты нелинейной опитки}


\section{Эффекты, связанные c поляризацией второго порядка}

\subsection{Возбуждение второй гармоники с использованием нарушенной симметрии III - V полупроводниковых метаповерхностей Фано}


\section{Эффекты, связанные c поляризацией третьего порядка}

\subsection{Усиленная генерация третьей гармоники в наночастицах кремния, обусловленная магнитным откликом}

\section{Использование диэлектрических метаповерхностей как широкополосный оптический частотный смеситель}


