\section{Основы линейной и нелинейной оптики.}

\subsection{Материальное уравнение линейной среды.}
\hspace*{2mm}
В основе взаимодействия света с веществом лежит процесс возбуждение частиц вещества внешней электромагнитной волной. Характер этого взаимодействия во многом определяется соотношением между величиной напряженности внешней электромагнитной волны и напряженностью внутриатомного поля. В случае, когда напряженность внешнего поля накачки много меньше внутриатомного поля, процесс называют линейным. Тогда зависимость поляризации волны в случае изотропной среды может быть представлена в виде:
\begin{equation}\label{1:liner}
P(E) = \chi^{(1)}E
\end{equation}
где $ \chi^{(1)}$ – линейная восприимчивость среды, являющаяся безразмерной величиной и зависящая только от свойств среды. На материальном уравнении  (\ref{1:liner}) базируется линейная оптика.  Но в случаях, когда напряженность внешнего поля оказывается  сравнимым с внутриатомным полем процесс становится нелинейным и должен быть описан другим уравнением. 

\subsection{Материальное уравнение нелинейной среды.} 
\hspace*{2mm}
Теория нелинейно-оптических явлений строится на основе материальных уравнений и уравнений Максвелла  \cite{achmanov2}. Уравнения Максвелла для диэлектрической нейтральной немагнитной среды имеют вид
\begin{equation}\label{1:maxvel}
rot\vec{E} = - \frac{ 1 }{ c}\frac{\partial \vec{H} }{\partial t}
\hspace{20mm}
rot\vec{H} =  \frac{ 1 }{ c}\frac{\partial \vec{D} }{\partial t}
\hspace{20mm}
div\vec{D} = 0
\hspace{20mm}
div\vec{H} = 0
\end{equation}где $ \vec{D} = \vec{E} + 4\pi \vec{P}$. Из уравнений Максвелла вытекает волновое уравнение
\begin{equation}\label{1:rot_maxvel}
rot(rot\vec{E}) + \frac{ 1 }{ c^2 }\frac{\partial^2 \vec{E} }{\partial t^2} = - \frac{ 4\pi }{ c^2 }\frac{\partial^2 \vec{P} }{\partial t^2}
\end{equation}которое в случае изотропной среды принимает вид
\begin{equation}\label{1:wave_eq}
\Delta\vec{E} - \frac{ 1 }{ c^2 }\frac{\partial^2 \vec{E} }{\partial t^2} =  \frac{ 4\pi }{ c^2 }\frac{\partial^2 \vec{P} }{\partial t^2}
\end{equation}
где $\vec{E}$ - напряженность электрического поля, а $\vec{P}$ - поляризация среды. Поляризация среды возникает под действием падающий световой волны и описывается материальным уравнением $\vec{P} = \vec{P}(\vec{E})$. Простейшее материальное уравнение нелинейной среды имеет вид
\begin{equation}\label{1:p_1}
P(E) = \chi^{(1)}E + \chi^{(2)}E^2 +\chi^{(3)}E^3\dots
\end{equation}
Коэффициенты $\chi^{(m)}, m \ge 2$ при  называются оптическими восприимчивостями m-го порядка, которые также зависят от свойств среды, но уже являются размерными величинами. Таким образом это уравнение есть нелинейная функция напряженности внешнего светового поля и описывает нелинейные оптические эффекты. 

\subsection{Основные процессы и эффекты нелинейной оптики}
Нелинейные оптические эффекты можно разделить на несколько групп (рис. \ref{classific}). 
\begin{figure}[h!]
	\centering
	\includegraphics[width=0.7\linewidth]{images/classific.png}
	\caption{Классификация нелинейных оптических эффектов \cite{mifi2015}}
	\label{classific}
\end{figure}
\\
К \textit{параметрическим явлениям} относятся: 
\begin{enumerate}
\item   Эффект Поккельса (явление возникновения двойного лучепреломления при наложении электрического поля, наблюдается только в кристаллах, не обладающих центром симметрии).
\item  Эффект Керра (явление изменения значения показателя преломления материала пропорционально квадрату напряженности приложенного электрического поля).
\end{enumerate}
К непараметрическим нелинейно-оптическим эффектам относятся такие как:
\begin{enumerate}
\item  Эффекты генерации высших оптических гармоник.
\item  Самовоздействие интенсивного светового пучка в нелинейных материалах.
\item Оптический пробой среды (среда начинает резко поглощать излучение).
\end{enumerate}
Эффекты, \textit{обусловленные мнимой составляющей нелинейных восприимчивостей}:
\begin{enumerate}
\item  Многофотонные процессы (в элементарном акте взаимодействия света с атомом участвует несколько фотонов).
\item  Вынужденное комбинационное рассеяние света.
\item Вынужденное рассеяние Мандельштама – Бриллюэна  (мощное световое излучение возбуждает в среде когерентные колебания молекул).
\end{enumerate}


\subsection*{Генерация второй оптической гармоники} 
\hspace{2mm}
Обсудим подробнее эффект удвоения частоты света в кристалле — генерацию второй оптической гармоники \cite{achmanov2}. Данный эффект состоит в том, что под действием мощного лазерного излучения в нелинейном кристалле возникает излучение на удвоенной частоте.
\\
\hspace{2mm}
Предположим, что на квадратично-нелинейную среду воздействует монохроматическое поле, тогда нелинейная поляризация второго порядка будет пропорциональна полю во второй степени:
\begin{equation}\label{thg1}
\vec{P}(2\omega) = \chi^{(2)}(2\omega = \omega + \omega):\vec{E}(\omega)\vec{E} (\omega)
\end{equation}

\begin{figure}[h]
	\centering
	\includegraphics[width=0.6\linewidth]{images/shg.png}
	\caption{Схематическое представление нелинейно-оптических процессов второго порядка. \cite{mifi2015}}
	\label{sghPictr}
\end{figure}

\subsection*{Генерация третьей оптической гармоники} 
Пусть на кубически-нелинейную среду воздействует монохроматическое поле. Тогда нелинейная поляризация второго порядка будет пропорциональна полю уже в третей степени:
\begin{equation}\label{thg2}
\vec{P}(3\omega) = \chi^{(3)}(3\omega = \omega + \omega + \omega):\vec{E}(\omega)\vec{E} (\omega)\vec{E} (\omega)
\end{equation}
Но обычно нелинейный член третьего порядка много меньше нелинейного члена второго порядка, соответственно  эффект генерации третей гармоники в кубически-нелинейной среде очень мал. На практике третью гармонику лазерного излучения получают путем последовательного удвоения частоты $\omega +  \omega \rightarrow 2\omega$, а затем сложения волн первой и второй гармоники: $\omega +  2\omega \rightarrow 3\omega$ в квадратично-нелинейной среде.
\begin{figure}[h]
	\centering
	\includegraphics[width=0.4\linewidth]{images/thg.png}
	\caption{Поляризация среды третьего порядка \textbf{(a)}. Схематичное представление процесса генерации третей гармоники \textbf{(b)}) \cite{mifi2015}}
	\label{thg12}
\end{figure}

\section{Оптические резонансы диэлектрических и не диэлектрических наноструктур}
\hspace*{2mm}
Наноразмерная оптика первоначально была связана с плазмонными структурами, сделанными из металлов, таких как золото или серебро. Основной проблемой наноплазмоники являются большие потери на нагрев металла и трудности в производстве. Недавние разработки в области наноразмерной оптической физики привели к появлению новой области -  нанофотоники. Исследования в этой области направленны на управление светом посредством резонансов Ми в диэлектрических и полупроводниковых наночастицах с высокими показателями преломления \cite{kuznetsov2016optically}. Такие частицы предлагают уникальные возможности для уменьшения диссипативных потерь и резонансного усиления как электрического, так и магнитного полей. 
\\
\hspace*{2mm}
Сосуществование сильных электрических и магнитных резонансов, их интерференция в диэлектрических наночастицах открывают совершенно новые функциональные возможности, которые можно реализовать в простых геометриях. 

\subsection*{Ми резонансы в субволновых частицах}

\begin{figure}[h!]
	\centering
	\includegraphics[width=0.7\linewidth]{images/fig1.png}
	\caption{\textbf{(a)}Эффективность рассеяния (безразмерное отношение сечения рассеяния к геометрическому сечению частицы) в зависимости от диэлектрической проницаемости $\varepsilon$ (частицы без потерь, q = 0,5) для металлических ($\varepsilon$  < 0) и диэлектрических ( $\varepsilon$ > 0) материалов. Сокращения для резонансов: ed - электрический диполь; eq - электрический квадруполь; md - магнитный диполь; mq - магнитный квадруполь. \textbf{(b)} Эффективность рассеяния диэлектрической частицы без потерь (цветная шкала справа) как функция показателя преломления n и параметра q. \textbf{(c)} Иллюстрация структур электрического и магнитного поля для различных электрических и магнитных резонансов, поддерживаемых сферической диэлектрической частицей. \cite{kuznetsov2016optically}}
	\label{fig1}
\end{figure}
\hspace*{2mm}
Чтобы проиллюстрировать фундаментальные свойства рассеяния света наночастицами, рассмотрим случай сферической частицы, освещенной плоской волной, для которой существует точное аналитическое решение уравнений Максвелла. Согласно теории Ми \cite{absorbScattLight}, металлические и диэлектрические сферические частицы могут обладать сильными резонансами рассеяния (рис. \ref{fig1}a). В случае немагнитных материалов их свойства рассеяния зависят только от двух параметров: диэлектрической проницаемости $\varepsilon$  и размерного параметра q ($q = 2\pi R/\lambda$). Различие между металлическими и диэлектрическими частицами заключается в знаке диэлектрической проницаемости, которая является отрицательной для металлов и положительной для диэлектриков. Небольшие металлические сферы (q < 1) создают только локализованные поверхностные плазмонные резонансы электрического типа:  дипольные, квадрупольные и т.д., в то время как их магнитный отклик остается практически незначительным из-за малого поля внутри сферы (рис. \ref{fig1}a).  
\\
\hspace*{2mm}
Чтобы воспроизвести "ощутимый" магнитный отклик от металлических структур, геометрия частицы должна быть другой. Например, резонатор с вырезанным кольцом  работает аналогично LC-контуру, в котором усиление магнитного поля происходит в центре него. Для диэлектрических наночастиц наблюдается как электрический, так и магнитный отклики (рис. \ref{fig1}a). Резонансный магнитно-дипольный отклик является результатом взаимодейсвия входящего света с круговыми токами смещения электрического поля частицы. Это происходит, когда длина волны внутри частицы становится сопоставимой с диаметром частицы $2R \approx \lambda/n$ (рис. \ref{fig1}b) \cite{kuznetsov2016optically}. 
\\
\hspace*{2mm}
Полевая структура четырех основных резонансных мод диэлектрических частицах с высоким показателем преломления показана на рисунке \ref{fig1}c. На длине волны магнитного резонанса возбужденная магнитно-дипольная мода диэлектрической сферы с высоким показателем преломления может вносить основной вклад в эффективность рассеяния, превосходя моды других мультиполей на порядки величины.
\\
\hspace*{2mm}
Из теории Ми следует, что максимально достижимая эффективность рассеяния для конкретного мультипольного возбуждения субволновой частицы зависит только от резонансной частоты, а не от типа материала \cite{schuller2009general}. Это говорит о том, что многие плазмонные эффекты, наблюдаемые при рассеянии света металлическими наночастицами, могут быть реализованы с помощью диэлектрических наночастиц с высоким показателем преломления. На рисунке \ref{fig1}b показано масштабирование различных резонансов по отношению к показателю преломления n. При n > 2 все основные мультиполи хорошо определены, и их спектральные положения соответствуют фиксированному отношению длины волны внутри частицы к ее диаметру. Эффективность рассеяния всех мультиполей также увеличивается с ростом n.

\subsection*{Наблюдение оптических магнитных резонансов в диэлектрических наночастицах.}
\hspace*{2mm}
Сильные оптически индуцированные магнитные дипольные резонансы в диэлектрических наночастицах с высоким показателем преломления могут быть достигнуты не только для сфер, но и для сфероидов \cite{articleDirVi}, дисков и цилиндров, колец и многих других геометрий. Это обеспечивает возможности для создания разнообразных полностью диэлектрических наноструктур с желаемыми спектральными положениями резонансов. Спектральные положения как электрического, так и магнитного дипольного резонансов могут настраиваться независимо путем изменения формы частиц.
\\
\hspace*{2mm}
Кремниевые (Si) наносферы с размерами от 100 нм до 300 нм поддерживают сильные магнитные и электрические дипольные резонансы в видимой и ближней ИК областях спектра (рис. \ref{fig2}c). 
 \begin{figure}[h]
	\centering
	\includegraphics[width=0.5\linewidth]{images/graph1.png}
	\caption{\textbf{(a, b)} Изображения в темнопольном оптическом микроскопе (слева), изображения на сканирующем электронном микроскопе (SEM) (справа)  сферических наночастиц Si с приблизительными диаметрами 100 нм \textbf{(a)} 140 нм \textbf{(b)} \cite{kuznetsov2012luk}. \textbf{(c)} Результат моделирования оптического магнитного отклика наночастиц кремния  диаметром 100 нм. (синий график) и 140 нм. (оранжевый график).  Сокращения для резонансов: ed - электрический диполь; md - магнитный диполь}
	\label{fig2}
\end{figure}
Экспериментальная демонстрация электрических и магнитных дипольных резонансов в видимом диопазоне волн впервые была представлена для сферических наночастиц Si, полученных с помощью фемтосекундной лазерной абляции на кремниевых и стеклянных подложках \cite{kuznetsov2012luk}. 
\\
\hspace*{2mm}
Различные цвета, наблюдаемые при помощи темнопольного микроскопа (рис. \ref{fig2}a, b), соответствуют магнитным дипольным резонансам почти идеальных сферических наночастиц Si с размерами от 100 нм до 200 нм.  Помимо кремния, полупроводники группы IV и группы III-V с показателями преломления выше 2 могут иметь аналогичные оптические свойства в зависимости от их поглощения и показателя преломления в конкретном диапазоне длин волн. Например, магнитные и электрические дипольные резонансы недавно были экспериментально обнаружены в нанодисках арсенида галлия (GaAs) в видимом спектре \cite{person2013demonstration}.


\section{Нелинейные оптические резонансны диэлектрических наноструктур}
\hspace*{2mm}
Усиление в ближнем поле электрического и магнитного отклика в полностью диэлектрических наноструктурах может привести к новым нелинейным эффектам. В частности, генерация второй и третьей гармоник (ГВГ и ГТГ), самоиндукция света и комбинационное рассеяние. Известно, что плазмонные резонансы, которые усиливают локальные электрические поля, усиливают нелинейно-оптические эффекты в металлических наноструктурах. В отличие от плазмоники, при резонансе диэлектрических наночастиц с высоким показателем преломления генерируется большее количество различных мод, которые в последствии могут привести к резкому увлечению интенсивности.

\subsection*{Генерация третьей гармоники в наночастицах кремния, обусловленная магнитным откликом}
\hspace*{2mm}
Усиленная ГТГ от нанодисков Si, демонстрирующих как электрический, так и магнитный дипольный резонансы, наблюдалась экспериментально \cite{shcherbakov2014enhanced}. 
\begin{figure}[h!]
	\centering
	\includegraphics[width=1\linewidth]{images/fig4.png}
	\caption{Генерация третей гармоники от нанодисков Si, усиленная магнитным резонансом . \textbf{(а)} Изображение матрицы кремниевых нанодисков с d = 360 нм, h = 260 нм и p = 2,85 мкм. \textbf{(b)} Экспериментальный нормированный спектр пропускания образца (черный) со спектром импульса накачки, обозначенным красной областью. ED  обозначает положение электрического дипольного резонанса, а MD - положение магнитного дипольного резонанса. \textbf{(c) }Рассчитанные спектры сечения рассеяния нанодисков (черный), разложенных на электрический дипольный (красные) и магнитный дипольный (синие) вклады с соответствующими распределениями электрического поля. \textbf{(d)} Микроскопическое изображение образца. \cite{shcherbakov2014enhanced}}
	\label{nonliner:nanodisks1}
\end{figure}
\hspace*{2mm}
Диски освещались интенсивным фемтосекундным лазерным импульсом с частотой $\omega$, близкой к магнитно-дипольному резонансу . В результате высокой восприимчивости кремния третьего порядка $\chi^3$ передаваемый сигнал содержал импульсы утроенной основной частоты $3\omega$. Поскольку сигнал третьей гармоники (ТГ) пропорционален кубу локальной напряженности поля, разумно ожидать значительного усиления процесса ТГ в нанодисках с их магнитными резонансами, возбуждаемыми фундаментальной волной. (рис. \ref{nonliner:nanodisks}). 
\begin{figure}[h!]
    \centering
	\includegraphics[width=0.8\linewidth]{images/fig3.png}
	\caption{\textbf{(a)}. ГТГ-спектроскопия массива нанодисков Si. Отрицательный логарифм нормированного спектра пропускания образца с p = 0,8 мкм, h = 220 нм и d = 0,5 мкм показан серой областью, указывающей на резонанс при 1,24 мкм. Спектр ГТГ образца, нормированный по спектру подложки, показан фиолетовыми точками. На вставке показано изображение фрагмента образца. Масштабная линейка  500 нм. \textbf{(b)} Моделируемый спектр пропускания основной длины волны через диски и подложку, хорошо согласующийся с экспериментом. Синие и красные кривые дают величину электрических и магнитных дипольных моментов (соответственно). \cite{shcherbakov2014enhanced}}
	\label{nonliner:nanodisks}
\end{figure}
\hspace*{2mm}
Локализация поля в магнитном резонансе приводит к увеличению интенсивности гармоник на два порядка по отношению к неструктурированному объемному Si, при этом эффективность преобразования ограничивается только двухфотонным поглощением в подложке \cite{shcherbakov2014enhanced}. 


\section{Метаповерхности на основе резонансных металлических наноструктур.}
\hspace*{2mm}
Когда свет взаимодействует с металлическими наноструктурами, он  воздействуют на свободные электроны вблизи поверхности металла, образуя плазмоны. Электромагнитные резонансы, связанные с этими поверхностными плазмонами, зависят от геометрии наноструктуры. Результирующие резонансные электромагнитные поля позволяют значительно усилить слабые нелинейные процессы, которые линейно зависят от локального поля.
\\
\hspace*{2mm}
Плазмонные возбуждения могут усиливать нелинейно-оптические эффекты несколькими способами. Во-первых, связь света с поверхностными плазмонами может приводить к усилению локальных электромагнитных полей \cite{novotny2011antennas}, которые значительно усиливают оптические процессы. Ярким примером является увеличение на несколько порядков процесса  комбинационного рассеяния. Во-вторых, плазмонные возбуждения могут быть чрезвычайно чувствительными к диэлектрическим свойствам окружающей среды. Это является основой для плазмонных датчиков: незначительные изменения показателя преломления вблизи поверхности металла приводят к значительным модификациям плазмонного резонанса \cite{homola2008surface}.
\\
\hspace*{2mm}
Отклик второго порядка обычно приводит к эффектам смешения волн, которые приводят к преобразованию частоты. Наиболее распространенным примером является генерация второй гармоники (ГВГ). Поверхностно-модулируемая вторая гармоника была впервые изучена с использованием поверхности серебра \cite{chen1981surface}. Сигнал второй гармоники был рассеянный, но довольно сильный. Результаты показали, что сигналы являются некогерентными и усиливаются резонансами наноразмерных поверхностных элементов. Дальнейшее подтверждение роли таких резонансов было получено из экспериментов генерации второй гармоники на пленках с металлическими вкраплениями \cite{wokaun1981surface}. Эти и дополнительные исследования показывают, что сигнал второй гармоники зависит от поляризации основного поля.
\\
Новые частоты возникают также из-за нелинейностей третьего порядка. Отклик третьего порядка содержит члены на падающих частотах - это называется оптическим эффектом Керра и приводит к нелинейным модификациям показателя преломления. Комбинируя нелинейности Керра с оптическими системами с обратной связью, можно получить бистабильность, когда один входной сигнал допускает два возможных выхода. Для оптических пучков с конечным поперечным размером дифракционные и нелинейные эффекты могут уравновешивать друг друга, создавая оптические солитоны.

\subsection*{Структурированные плазмонные поверхности для усиления нелинейных эффектов.}
\hspace*{2mm}
Рассмотрим пример нецентросимметричной поверхности  из массива L-образных наночастиц. Была исследована  способность подобных массивы L-образных частиц золота к генерации второй гармоники (рис. \ref{mettals}а). Было обнаружено, что эффективность сильно зависит от упорядочения частиц в решетке, причем самые сильные сигналы возникают, когда основная длина волны находится в плазмонном резонансе структуры. 
\\
Резонаторы с расщепленным кольцом \cite{linden2012collective} (рис. \ref{mettals}b) имеют симметрию, аналогичную L-образным наночастицам. Плазмонные резонансы можно интерпретировать как имеющие электрический или магнитный характер, хотя сами составляющие материалы являются немагнитными. 
\begin{figure}[h!]
	\centering
	\includegraphics[width=0.9\linewidth]{images/mettals.png}
	\caption{Примеры структурированных металлических поверхностей для нелинейной плазмоники. \textbf{(a , b)}, Некоторые из основных случаев, исследованных для ГВГ: \textbf{(а)} L-образные частицы золота; \textbf{(b) }золотая структура; \textbf{(c) },  Более совершенная структура для усиления нелинейных эффектов: золотая решетка для улучшения ЧСВ-сигала (ЧСВ(WM) - частотное смешение волн). Синяя часть представляет собою падающие поля E1 и E2 на соответствующих частотах $\omega_1$ и $\omega_2$, а также сгенерированный сигнал на частоте $4\omega_{WM}$. Желтая часть представляет собой золотую решетку с указанными критическими размерами. \cite{genevet2010large}, \cite{linden2012collective}}
	\label{mettals}
\end{figure}
\hspace*{2mm}
Генерация второй гармоники также обсуждалась и для центросимметричных образцов, включая массивы наночастиц и наноапертур. Во всех случаях наиболее сильная вторая гармоника была обнаружена, когда основной пучок распространялся под косым углом падения. Вот, например, в недавней работе были продемонстрированы более совершенные плазмонные поверхности с улучшенными нелинейными свойствами. Кажущаяся центросимметрия диэлектрических решеток была нарушена для ГВГ из-за освещения металла под косым углом \cite{genevet2010large} (Рис. \ref{mettals}c). 
\\
\hspace*{2mm}
Наноструктурированные металлические поверхности также используются для усиления отклика известных нелинейных материалов. Один интересный образец состоял из массивов коаксиальных дырок в золотой пленке толщиной 70 нм на подложке из арсенида галлия (GaAs)  \cite{fan2006second}. Структуры демонстрировали отклик второй гармоники вблизи среза коаксиальной волноводной моды.
\section{Метаповерхности на основе резонансных диэлектрических элементов.}
\hspace*{2mm}
Резонансные диэлектрические наночастицы также могут быть использованы для создания плоских однослойных решеток, известных как метаповерхности \cite{yu2014flat}. Использование метаповерхностей позволяет наблюдать различные резонансы  электрического и магнитно-оптического отклика в сочетании с низкими потерями тонкослойных структур. Одной из основных функций метаповерхностей является контроль фазы отраженного и прошедшего света. Идея использования диэлектриков, структурированных в масштабе длины волны для управления волновым фронтом, активно обсуждалась раньше \cite{lalanne1999design}. Другой подход к достижению фазового контроля был предложен в начале 2000-х годов. Он опирается на металлические или диэлектрические субволновые решетки. Недавняя работа, основанная на этом принципе, продемонстрировала плоские кремниевые линзы и аксиконы, работающие на видимых длинах волн (рис.  \ref{nonliner:matasurf1}a) \cite{lin2014dielectric}. Новым подходом к созданию диэлектрических метаповерхностей является использование электрических и магнитных дипольных резонансов диэлектрических наночастиц для контроля фазы подающего света \cite{shalaev2015high}. Каждый из диполей способен сдвигать фазу от 0 до $\pi$ вблизи резонанса. Объединение отклика обоих диполей на одной и той же длине волны позволяет достичь полного перекрытие резонансов \cite{decker2015high}. 
 \begin{figure}[h!]
	\centering
	\includegraphics[width=0.8\linewidth]{images/fig5.png}
	\caption{ Нанофотонные устройства на основе полупроводниковых наноантенн. \textbf{(a)} Изображение фотодетектора с использованием оптически резонансной нанопроволоки Ge с асимметричными металлическими контактами. \textbf{(b)} Спектры фототока отдельных нанопроволок Ge с радиусами 10 нм (черный), 25 нм (синий) и 110 нм (красный). \textbf{(c)} Двумерный график рассчитанной эффективности поглощения как функции длины волны и радиуса нанопроволоки. Пунктирные серые линии указывают радиусы, для которых спектральный фототок показан в \textbf{(b). (d)} Изображение матрицы резонаторов Si, которая служит антибликовым покрытием на подложке Si. \textbf{(e)} Измерение зеркального отражения при 632 нм в зависимости от угла падения для непокрытой пластины Si (черного цвета), с антиотражающем покрытием Si3N4 толщиной 60 нм (красного цвета) и матрицы резонаторов Ми, как показано на \textbf{(d)} покрытой слоем $Si_3N_4$ толщиной 60 нм (синий) \cite{spinelli2012broadband}. }
	\label{nonliner:matasurf1}
\end{figure}
\\
\hspace*{2mm}
Недавно были продемонстрированы экспериментально метаповерхности видимого диапазона на основе кремния с резонансным пропусканием более 85\%. Эти устройства имеют толщину всего 130 нм, что соответствует менее чем одной пятой длины волны в свободном пространстве, и способны контролировать волновой фронт света и выполнять отклонение луча с эффективностью передачи, близкой к 50\% (рис. \ref{nonliner:matasurf1}, b и c). Аналогичные концепции были также использованы для достижения эффективной генерации вихревого пучка (рис. \ref{nonliner:matasurf1}d). 
\\
\hspace*{2mm}
Диэлектрические поверхности также могут быть использованы в качестве практически идеальных отражателей.  Эффект отражения возникает в результате когерентного взаимодействия магнитных или электрических диполей, возбуждаемых внешним излучением в каждой наночастице. Недавний эксперимент продемонстрировал очень высокий коэффициент отражения (до 99,7\%) в ближней ИК-области спектра от плотного массива наночастиц Si \cite{moitra2015large}. Такое поведение с высоким коэффициентом отражения концептуально аналогично тому, которое наблюдалось ранее в высококонтрастных решетках с субволновой длиной. 
\\
\hspace*{2mm}
Другим примером является обобщенный эффект Брюстера. В отличие от обычного эффекта Брюстера, который ограничен p-поляризованным падением и углами выше $45^\circ$ (угол Брюстера ниже $45^\circ$ всегда сопровождается полным внутренним отражением под большими углами), обобщенный эффект Брюстера может быть достигнут для любой поляризации (как p, так и s) и любого угла падения (как ниже, так и выше $45^\circ$ без полного внутреннего отражения) (рис. \ref{nonliner:matasurf1}e). Оно возникает в результате интерференции электрических и магнитных резонансов в решетке и подавления их рассеяния в направлении отражения.

\subsection*{Возбуждение второй гармоники с использованием нарушенной симметрии III - V полупроводниковых метаповерхностей Фано}

\hspace*{2mm}
Схема Фано-резонансной метаповерхности показана на рисунке  \ref{nonliner:matasurf}а. Форма нанорезонатора приводит к смешению мод между поперечными и продольными дипольными модами света, что приводит к резонансу Фано с высокими значениями добротности. Эти резонансы могут наблюдаться как в спектре  пропускания так и в спектре отражения.
\begin{figure}[h!]
    \centering
	\includegraphics[width=0.5\linewidth]{images/fig6.png}
	\caption{Несимметричная метаповерхность GaAs используется для создания резонансов Фано с высоким Q. \textbf{(а) }Схема фанорезонансной структуры метаповерхности, основанная на кубе со стороной $\approx 300$ нм, который имеет угловой вырез. Верхняя часть вставки - изображение метаповерхности Фано. Вставка снизу: изображение метаповерхности, полученное под углом $20^\circ$. Масштабная линейка соответствует 1 мкм. \textbf{(b)} Спектр отражательной способности метаповерхности Фано, который обладает высоким Q-резонансом с шириной спектра $\approx$ 2 нм. Вставка представляет собой увеличенную область спектра отражения вблизи резонанса Фано. \cite{vabishchevich2018enhanced}}
	\label{nonliner:matasurf}
\end{figure}
Резкий провал в спектрах отражения на рис. \ref{nonliner:matasurf}b при $\lambda_{F} \sim0,99$  мкм соответствует возбуждению резонанса Фано со спектральной шириной $\Delta\lambda_{F} \sim 2$ нм (добротность$\lambda_{F} /\Delta\lambda \sim 500$)). Численные расчеты показывают, что в идеальной наноструктуре усиление электрического поля внутри нанорезонатора при резонансе Фано достигает значения $|E|^2/|E_{0}|^2 \sim 500$. На рис. \ref{nonliner:matasurf2}а показана интенсивность сигнала второй гармоники, генерируемого  ассиметричной мета-поверхностью GaAs. Спектр интенсивности ГВГ имеет узкий пик при $\sim$ 0,99 мкм, который соответствует положению резонанса Фано в линейном спектре. Из-за локализации электромагнитного поля и увеличения его интенсивности, внутри нанорезонатора сигнал ГВГ  (когда лазер накачки настроен на резонанс Фано) в  $\sim$ 300 раз сильнее по сравнению с нерезонансным случаем. На рисунке \ref{nonliner:matasurf2}a показан сценарий модификации нормализованного спектра ГВГ, когда длина волны несущей накачки отрегулирована относительно резонансной длины волны. Когда $\Delta\lambda_{F} = -10$ нм, перекрытия с резонансом Фано не происходит, и форма генерируемой второй гармоники близка к гауссовой. На рис. \ref{nonliner:matasurf2}b показано сравнение спектра второй гармоники, полученного на метаповерхности GaAs при накачке на длине волны Фано-резонанса и ВГ, генерируемая в подложке GaAs. 
 \begin{figure}[h!]
	\centering
	\includegraphics[width=0.7\linewidth]{images/fig7.png}
	\caption{\textbf{(a): }Генерация второй гармоники (ГВГ) на резонансной поверхности Фано. Точки: экспериментальная спектральная зависимость интенсивности ГВГ, демонстрирующая резонансно усиленную ГВГ на длинах волн резонанса Фано. Фиолетовый: линейный спектр отражения мета-поверхности Фано. Вставка: вид сверху образца РЭМ. \textbf{(b):} Сужение спектров второй гармоники, генерируемых мета-поверхностью Фано (черная линия), по сравнению со второй гармоникой, генерируемым непокрытой подложкой GaAs (серая линия) \cite{vabishchevich2018enhanced}}
	\label{nonliner:matasurf2}
\end{figure}

\subsection*{Использование диэлектрических метаповерхностей как широкополосного преобразователя частот}
\textit{Преобразователь частот} - это нелинейное устройство, которое объединяет электромагнитные волны для создания волн на новых частотах. Эти устройства широко используются в  современных радиочастотных технологиях и в обработке микроволновых сигналов. Разработка универсальных частотных преобразователей для оптических частот остается сложной задачей: такие устройства обычно работают на слабых нелинейных оптических процессах и должны удовлетворять условиям согласования фаз.
\\
\hspace*{2mm}
Рассмотрим полупроводниковую метаповерхность на основе GaAs.  Преобразователь частот на ее основе способен одновременно генерировать одиннадцать новых частот, охватывающих ультрафиолет и инфракрасный диапазоны. Нелинейности четного и нечетного порядка GaAs позволяют наблюдать генерацию \textit{второй гармоники}, \textit{третьей гармоники} и \textit{четвертой гармоники}, \textit{генерацию суммарной частоты}, \textit{двухфотонную фотолюминесценцию}, вызванную поглощением,\textit{ четырехволновое и шестиволновое смешение}. Одновременному возникновению этих семи нелинейных процессов помогают комбинированные эффекты сильных внутренних нелинейностей материала, усиленных электромагнитных полей и ослабленных требований к фазовому согласованию.
\\
\hspace*{2mm}
На рис. \ref{mixerPictr1}а приведена схема генерации нелинейной частоты метаповерхностью GaAs, накачиваемой двумя лазерными лучами. Метаповерхность состоит из периодической квадратной матрицы наноцилиндров диаметром $\sim 400$нм. Каждый наноцилиндр состоит из трех слоев: верхний надодиск из $SiO_x (\sim 300 $нм), средний нанодиск из GaAs толщиной $\sim 450$нм, в котором локализуется электромагнитное поле, и нижний слой из $(Al_xGa_{1 - x})_20_3 \sim 400$нм с низким показателем преломления  для того чтобы отделить весь цилиндр от подложки.
\\
\hspace*{2mm}
Резонансное усиление частот достигается за счет одновременного  возбуждения магнитных дипольных (MD) и электрических дипольных (ED) Ми-резонансов наноцилиндра GaAs \cite{liu2016iii}. В измеренном спектре отражатения (рис. \ref{mixerPictr1}b) наблюдались максимумы при $\lambda_{ED} \sim 1,25$мкм и $\lambda_{MD} \sim 1,5$мкм, которые соответствуют возбуждению резонансов ED и MD. Это  подтверждается выполнением мультипольного разложения полей рассеяния, а также моделированием профилей электрического поля (показано на вставках) для двух длин волн, которые соответствуют максимальному усилению электромагнитного поля внутри нанорезонаторов (1,246 мкм и 1,535 мкм).
\begin{figure}[h!]
	\centering
	\includegraphics[width=0.7\linewidth]{images/mixer.png}
	\caption{Генерация новых частот с помощью метаповерхностью из  GaAs. Схема метаповерхности, состоящей из квадратной матрицы субволновых диэлектрических резонаторов. \textbf{(a):} Два фемтосекундных ближних ИК-импульса накачивают структуру, и одновременно генерируется множество новых частот. Изображение  поверхности GaAs показано на вставке  под углом 60 $^\circ$. Масштабная линейка соответствует 3 мкм. Нижняя вставка: схематические энергетические диаграммы семи нелинейных оптических процессов, которые происходят одновременно в эксперементе: генерация второй гармоники (ГВГ), генерация третьей гармоники (ГТГ), генерация четвертой гармоники (ГЧГ), генерация суммарной частоты (ГСЧ) фотолюминесценция, индуцированная двухфотонным поглощением (TPA PL), четырехволновое смешение (ЧВС) и шестиволновое смешение (ШВС). \textbf{b} Измеренные (сплошная линия) и численно смоделированные (пунктирная линия) спектры отражения метаповерхности с двумя поперечными локальными распределениями электрического поля на длинах волн 1,246 мкм и 1,535 мкм, которые соответствуют максимальным усилениям электромагнитного поля внутри нанодиска GaAs. \textbf{(c)} Спектры второй, третьей и четвертой гармоник, когда импульсы накачки $\lambda_1 \sim $1,57 мкм используются для возбуждения метамиксера. Втсавка - это увеличение четвертой гармоники \cite{liu2018all}}
	\label{mixerPictr1}
\end{figure}
\\
\hspace*{2mm}
\textbf{Генерация гармоник в однолучевых экспериментах}. При накачке одним фемтосекундным лучом с длиной волны в близи MD резонанс ($\lambda_1 \sim$ 1,57 мкм)  генерируются вторая ($\lambda_{SHG} \sim$ 785 нм), третья ($\lambda_{THG} \sim$ 523 нм) и четвертая ($\lambda_{FGH} \sim$ 393 нм) гармоники (рис. \ref{mixerPictr1}с) . Излучение с центром при $\lambda_{PL} \sim$ 870 нм соответствует GaAs-фотолюминесценции (PL), возникающей при двухфотонном поглощении (TPA) накачки. Наблюдаемые гармоники лежат выше энергии запрещенной зоны GaAs и поэтому значительная часть их излучения поглощается.
\\
\hspace*{2mm}
\textbf{Смешивание частот в двухлучевых экспериментах}.  Когда два луча накачки (первый - $\lambda_1 \sim$ 1,57 мкм, второй - $\lambda_2 \sim$ 1,24 мкм)  совпадают во времени, мы наблюдаем одиннадцать спектральных пиков в диапазоне от $\sim$ 380 нм до  $\sim$ 1000 нм (рис. \ref{mixerPictr2}а).  Сгенерированные сигналы можно условно разделить на две группы:
\\
Первая группа (обозначенная синими метками) соответствует процессам генерации гармоник, а также PL, возникающей при двухфотонном поглощении. Каждый из процессов в этой группе опирается только на один луч накачки. 
\\
Напротив, вторая группа (обозначенная красными метками) соответствует процессам смешивания частот, которые требуют  наличия обоих импульсов накачки. Пять сигналов смешения частот включают в себя: генерацию суммарной частоты $(\omega_1 + \omega_2)$ при $\lambda_{SHG} \sim$ 689 нм; три типа четырехволнового смешения $(2\omega_1 + \omega_2, 2\omega_2 + \omega_1, 2\omega_2 - \omega_1)$при $\sim$ 1000 нм, $\sim$ 472 нм и $\sim$ 434 нм соответственно; и шестиволновое смешение $(4\omega_1 - \omega_2)$ при $\lambda_{SWM} \sim$ 577 нм. Отметим, что амплитуда пика ЧВС $(2\omega_2 - \omega_1)$ как минимум в десять раз выше, чем у любого другого нелинейного процесса. Это объясняется гораздо меньшим поглощением ниже запрещенной зоны у  GaAs в отличие от сильного затухания, испытываемого другими сигналами. В целом в спектре содержат одновременные вклады семи нелинейно-оптических процессов .

\begin{figure}[h!]
	\centering
	\includegraphics[width=0.7\linewidth]{images/mixer1.png}
	\caption{\textbf{(a)} Смешивание частот в метаповерхности GaAs. Спектр, демонстрирующий одиннадцать нелинейно генерируемых пиков, возникающих в результате семи различных нелинейных процессов, когда два оптических луча с  $\lambda_2 \sim$ 1,24 мкм и  $\lambda_1 \sim$ 1,57 мкм используются для одновременной накачки метаповерхности GaAs. Синие метки указывают на процессы генерации гармоник и фотолюминесценции, возникающие в результате двухфотонного поглощения, для каждого из которых требуется только один луч накачки. Красные метки указывают на частоты, которое включает оба луча насоса. \cite{liu2018all}}
	\label{mixerPictr2}
\end{figure}
\hspace*{2mm}
Экспериментальная демонстрация семи разных нелинейных оптических процессов, происходящих одновременно в GaAs метаповерхности может быть использована для реализации ультракомпактных оптических преобразователей. Видно, что нечетные нелинейные процессы высокого порядка могут генерировать гармоники высокого порядка, являющиеся основой генерации аттосекундных импульсов. Кроме того, эти метаповерхности могут быть оптимизированы для других нелинейных процессов смешивания, таких как генерация разностной частоты $\omega_{DFG} = \omega_1 - \omega_2$. Это позволило бы создавать фемтосекундные импульсы, охватывающие среднюю инфракрасную область спектра. 