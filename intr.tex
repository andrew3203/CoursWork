\\
\hspace*{2mm}
\underline{Нелинейная оптика} изучает процессы воздействия электромагнитной волны с веществами, у которых имеется нелинейный отклик на приложенное поле электромагнитное поле. В основе нелинейных процессов лежит  взаимодействие света с веществом на наноуровне, когда интенсивность световой волны достаточно велика и в элементарном акте поглощения начинает участвовать уже несколько фотонов. Также, при большой интенсивности световой волны  начинают возникать процессы, приводящие к изменению исходных свойств вещества. Прогресс нелинейной оптики во многом обязан развитию лазеров, которые могут генерировать свет с большой напряжённостью электрического поля, соизмеримой с напряжённостью микроскопического поля в атомах.
\\
\hspace*{2mm}
В данной работе будут изложены основные понятия нелинейной оптики, нелинейные эффекты в резонансных полупроводниковых структурах. Будут описаны последние открытия в этой области, такие как: возбуждение различных гармоник с использованием нарушенной симметрии III-V полупроводниковых метаповерхностей Фано, генерация гармоник с использованием кремниевых нанодисков, использование диэлектрических метаповерхностей в качестве широкополосного оптического частотного смесителя и другие явления. Так-же, будут приведены результаты моделирования эксперимента, по линейному отклику периодической наноструктуры из Германия.

