\\
\hspace*{2mm}
\underline{Нелинейная оптика} - это раздел физики, который изучает явления взаимодействия света, и вещества, которые протекают по разному в зависимости от интенсивности света. В этом разделе физики рассматриваются вещества, у которых имеется нелинейная зависимость вектора поляризации от вектора напряженности электрического поля световой волны. Для большинства, веществ такая нелинейность может наблюдаться лишь при очень высоких интенсивностях света. Такие интенсивности достигаются при помощи лазеров. Взаимодействие или процесс называются линейными, если их вероятность пропорциональна первой степени интенсивности излучения. Если же эта, степень больше единицы, они называются нелинейными.
\\
\hspace*{2mm}
В настоящее время нелинейная оптика является динамично развивающейся областью физики, которая помимо чисто теоретической системы знаний приобрела также существенную практическую составляющую, что позволило решить ряд важных прикладных и инженерных задач. Исследования нелинейных оптических процессов дали много приложений в физике и математике, способствовали развитию лазерной техники, спектроскопии, оптоволоконных линий связи, фотоники и оптоинформатики, а также нашли многочисленные применения в таких отраслях, как экология и медицина.
\\
\hspace*{2mm}
В данной работе будут изложены основные понятия нелинейной оптики, нелинейные эффекты в резонансных полупроводниковых структурах. Будут описаны последние открытия в этой области, такие как: возбуждение различных гармоник с использованием нарушенной симметрии III-V полупроводниковых метаповерхностей Фано, генерация гармоник с использованием кремниевых нанодисков, использование диэлектрических метаповерхностей в качестве широкополосного оптического частотного смесителя и другие явления. Так-же, будут приведены результаты моделирования эксперимента, по линейному отклику периодической наноструктуры из Германия.

