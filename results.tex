

Основные результаты курсовой работы могут быть сформулированный следующим образом:
\begin{enumerate}
\item Было дано краткое теоретическое описание основных явлений и процессов линейной и не линейной оптики. Дана классификация эти процессов. Описаны основные нелинейно-оптические эффекты в резонансных диэлектрических и металлических наночастицах и метаповерхностях. Описан ряд экспериментов, раскрывающих данные эффекты.  Дан срез современных знаний в нелинейной оптике. 
\item Построена модель линейного оптического отклика полупроводниковой метаповерхности на основе германия.  Получен резонанс  в  конце ИК-диапазоне на длине волны $ \lambda \approx  2000$нм. Выдвинуто предположения,  что при резонансе наблюдаться квазиволновая мода, возбуждение которой обусловлено геометрией метаповерхности.  
\end{enumerate}
